\documentclass[12pt,a4paper]{article}

% Some useful packages.
% \usepackage{amsmath}
% \usepackage{siunitx}
% \usepackage{graphicx}
% \usepackage{verbatim}
% \usepackage{mhchem}
% \usepackage{textcomp}
% \usepackage{setspace}

% Reduces margins substantially.
\usepackage{geometry}
\newgeometry{margin=2.5cm}

% Allows headers and footers.
\usepackage{fancyhdr}
\pagestyle{fancy}
% Get rid of annoying line under header.
\renewcommand{\headrulewidth}{0pt}

\lhead{}
\chead{}
\rhead{}

\newcommand{\ts}{\textsuperscript}
\newcommand{\HRule}{\rule{\linewidth}{0.5mm}}

% Turns off page numbering.
\pagenumbering{gobble}

\usepackage[backend=biber,style=numeric,sorting=none]{biblatex}
% \usepackage[backend=biber]{biblatex}
\addbibresource{references/references.bib} % note the .bib is required

\title{Investigating the effect of resolution on model skill and fidelity by use of an analogue}
\author{Mark Muetzelfeldt - UCL Department of Geography}

\date{16 January, 2015}

\begin{document}

\section*{Investigating the effect of resolution on model skill and fidelity by use of an analogue}

% Should be around 500 words.
% 20\ts{th} Century Reanalysis Project
% \SI{850}{hPa}

% Things to demonstrate:
% * Comprehensive understanding of subject area
% * Understanding of problems in the area
% * Ability to construct (and defend) and argument
% * Powers of analysis
% * Powers of expression

% Structure
% * Outline a research problem
% * What information is required to solve it?
% * Suggest a line of research

\subsection*{Research Problem}

Since the beginning of weather and climate modelling, there has been a drive to increase the
resolution of the models being used. This has tangible effects for weather modelling, delivering
more skilful forecasts \cite{buizza2003benefits} and predictions that are more localised. For
climate models, higher resolution means that smaller features, such as tropical cyclones, can be
better represented \cite{bengtsson1995hurricane}. As resolution increases, sub-gridcell processes such as cloud formation can also
move from being parameterised to being modelled directly. The rationale behind this is to produce
climate models of higher fidelity, i.e. climate models that more faithfully recreate the climatology
of the Earth.

So, what are the links between resolution, weather prediction skill and climate forecast fidelity?
Intuitively it makes sense that as resolution is increased, both the skill and fidelity should
increase, but is this the case? One way of investigating this would be to run a series of weather
and climate models at different resolutions, then comparing the results with observed data from the
Earth. This would, however, require massive computing resources, and use the Earth system, which is
difficult to monitor and varies slowly. A different, more tractable approach will be outlined below.

\subsection*{Line of Research}

The use of an analogue is suggested as a means of investigating the links between resolution, skill
and fidelity. The analogue would consist of a rotating, air-filled annulus, with a heated inner
cylinder. This would provide an easily observable system with interesting, chaotic dynamics that
reproduce some of the features of the atmosphere, such as baroclinic instability
\cite{castrejon2007baroclinic}. This analogue would be modelled using computer simulations at a
number of different resolutions, and the output of the model would be compared with the observed
behaviour of the analogue in two ways. First, the specific state of the simulation would be compared
with the state of the analogue, providing a measure of the predictive skill of the model. Second,
the overall statistical behaviour of the two would be compared.

This experimental set up would mean that large amounts of observational data could be collected in a
short period of time. With these data, the skill of the model could be determined, and after
suitable analysis of the data the fidelity of the model could also be worked out. Relationships
between the resolution of the model and its skill and fidelity could be investigated. For example,
as the model resolution increases, does the skill increase indefinitely or asymptotically, or does
it even decrease? Limitations of the system due to chaos should put an upper bound on its skill,
dependent on the Lyapunov exponent of the system. Similarly, how does the model's fidelity change
with different resolutions? One other interesting line of research would be looking at the
predictability of the system, looking for certain states or features that persisted for longer than
would be suggested than the Lyapunov exponent, and could also be accurately modelled. This would
have implications for seasonal predictions and blocking meteorological.

This experiment could have wide ranging outcomes, from analysis of the predictability of chaotic
systems to informing the decisions to spend large amounts of money on more expensive super-computers
to model the weather and climate at higher resolutions. 

\printbibliography[title={References}]

\end{document}
